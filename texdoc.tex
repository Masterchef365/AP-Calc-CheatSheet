\documentclass[12pt, english]{article}
\usepackage{mathtools}
\usepackage{amssymb}
\date{}
\title{ AP Calculus Cheat Sheet }
\begin{document}
\author{
	v1.5
}

\maketitle
\tableofcontents
\newpage

\section{Derivatives}
\subsection{Using limits to calculate derivatives}
	\begin{equation}
		\frac{d}{dx} f(x) = \lim_{h \to 0} \frac{f(x+h)-f(x)}{h}
	\end{equation}

\subsection{Utilities}
\text{Power rule:}
\begin{equation}
		\frac{d}{dx} x^n = nx^{n-1}
\end{equation}

\noindent
\text{Chain rule:}
\begin{equation}
		\frac{d}{dx} (f(x) \circ g(x)) = f'(g(x)) * g'(x)
\end{equation}

\begin{equation}
	\begin{aligned}
		\frac{d}{dx} c * f(x) = c * f'(x) \\
		\frac{d}{dx} c = 0 \\
		\frac{d}{dx} [f(x)+g(x)] = f'(x)+g'(x) \\
		\frac{d}{dx} e^x = e^x \\
		e = \lim_{b \to \infty} (1+\frac{1}{b})^b \\
		y-y_1 = m(x-x_1) \\
		\frac{d}{dx} \ln x = \frac{1}{x} \\
		\frac{d}{dx} cx = c \\
		\frac{d}{dx} \log_{b} x = \frac{1}{x} * \frac{1}{\ln b} \\
		\frac{d}{dx} f^{-1}(x) = \frac{1}{f'(f^{-1}(x))} \\
	\end{aligned}
\end{equation}

\section{Identities}
Trig Identities
\begin{equation}
	tan\ \theta = \frac{sin\ \theta}{cos\ \theta}
\end{equation}

Rules of logarithms 
\begin{equation}
	\begin{aligned}
		log_b (b^x) = x
		b^{(log_{b}\ x)} = x
		log_{b}\ a^n = n \cdot log_b\ a
	\end{aligned}
\end{equation}

Rules of exponents 
\begin{equation}
	\begin{aligned}
		\left(\frac{a}{b}\right)^n = \frac{a^n}{b^n} \\
		(x^a)^b = x^{ab} \\
		x^a \cdot x^b = x^{a + b} \\
		x^{-n} = \frac{1}{x^n} \\
		\frac{a}{x} \cdot \frac{b}{x} \cdot \frac{c}{x} \cdot ... = (a \cdot b \cdot c \cdot ...)^{1/x} \\
	\end{aligned}
\end{equation}

\section{Domain No-nos}
\begin{enumerate}
	\item Divide by 0
	\item $\sqrt{-x}$
	\item $log\ 0$ or $ln\ 0$ or $log_x\ 0$
	\item $log\ -x$
	\item $sin^{-1}(2)$
\end{enumerate}

\section{Factoring reminder}
\begin{equation}
	\frac{3450}{1080} = \frac{2*5*5*3*23}{2*5*2*3*3*2*3} = \frac{5*23}{2*3*2*3} = \frac{115}{36}
\end{equation}

\section{Limit solving examples}
\text{By factoring and cancelling}
(Only applies in limits)
\begin{equation}
	\frac{\sqrt{x}-4}{x-16} = \frac{\sqrt{x}-4}{(\sqrt{x}+4)(\sqrt{x}-4)} = \frac{1}{\sqrt{x}+4}
\end{equation}

\text{Reduction}
\begin{equation}
	\begin{split}
		\lim_{x \to \infty} \sqrt{9x^2 + x} - 3x \\
		\\
		\lim_{x \to \infty} \frac{\sqrt{9x^2 + x} - 3x}{1} \\
		\\
		\lim_{x \to \infty} \frac{\sqrt{9x^2 + x} - 3x}{1} * \frac{\sqrt{9x^2 + x} + 3x}{\sqrt{9x^2 + x} + 3x} \\
		\\
		\lim_{x \to \infty} \frac{9x^2 + x - 9x^2}{\sqrt{9x^2 + x} + 3x} \\
		\\
		\lim_{x \to \infty} \frac{x}{\sqrt{9x^2 + x} + 3x} * \frac{\frac{1}{x}}{\frac{1}{x}} \\
		\\
		\lim_{x \to \infty} \frac{1}{\frac{\sqrt{9x^2 + x}}{x} + 3} \\
		\\
		\lim_{x \to \infty} \frac{1}{\sqrt{\frac{9x^2 + x}{x^2}} + 3} \\
		\\
		\lim_{x \to \infty} \frac{1}{\sqrt{\frac{9x^2}{x^2} + \frac{x}{x^2}} + 3} \\
		\\
		\lim_{x \to \infty} \frac{1}{\sqrt{9 + \frac{1}{\infty}} + 3} \\
		\\
		\lim_{x \to \infty} \frac{1}{\sqrt{9 + 0} + 3} \\
		\\
		\lim_{x \to \infty} \frac{1}{3 + 3} \\
		\\
		\lim_{x \to \infty} \frac{1}{6} = \frac{1}{6} \\
	\end{split}
\end{equation}

\text{Solve by IVT}
\begin{equation}
	\begin{split}
		f(x) = 5x^4 - 3x^2 + 2x - 1 \\
		\text{show  } f(c) = 0 \text{ for some }
		c \in \mathbb{R} \\
		f(0) = -1 \text{ and } f(1) = 3 \\
		\text{ thus } \exists \text { } c \in (0,1) \text{ such that } f(c) = 0 \text{ by IVT } 
	\end{split}
\end{equation}

	Squeeze Theorem:
	Short notes because I'm too tired; Basically, use inequalities on both sides to 
	shrink down the space until you get it to a point that both are equal then you've solved it.

\end{document}
