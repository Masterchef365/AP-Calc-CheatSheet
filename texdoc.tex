\documentclass[12pt, english]{article}
\usepackage{mathtools}
\usepackage{amssymb}
\date{}
\title{ AP Calculus Cheat Sheet }
\begin{document}
\author{
	v1.6 (Condensed)
}

\maketitle

\section{Derivatives}
\subsection{Using limits to calculate derivatives}
	\begin{equation}
		\frac{d}{dx} f(x) = \lim_{h \to 0} \frac{f(x+h)-f(x)}{h}
	\end{equation}

\subsection{Utilities}
\text{Power rule:}
\begin{equation}
	\frac{d}{dx} x^n = nx^{n-1}
\end{equation}

\noindent
\text{Chain rule:}
\begin{equation}
	\frac{d}{dx} (f(x) \circ g(x)) = f'(g(x)) * g'(x)
\end{equation}

\noindent
\text{Derivatives of inverses}
\begin{equation}
	\frac{d}{dx} f^{-1}(x) = \frac{1}{f'(f^{-1}(x))} 
\end{equation}

\noindent
\text{Derivatives of logbase}
\begin{equation}
	\frac{d}{dx} \log_{b} x = \frac{1}{x} * \frac{1}{\ln b} 
\end{equation}

\noindent
\text{Derivative of a multiplier}
\begin{equation}
	\frac{d}{dx} cx = c 
\end{equation}

\noindent
\text{Derivative of natural log}
\begin{equation}
	\frac{d}{dx} \ln x = \frac{1}{x} 
\end{equation}

\noindent
\text{Derivative of multiplication}
\begin{equation}
	\frac{d}{dx} fg = f'g + fg'
\end{equation}

\noindent
\text{Quotient Rule}
\begin{equation}
	\frac{d}{dx} \left(\frac{f}{g}\right) = \frac{f'g - fg'}{g^2}
\end{equation}

\noindent
\text{Tangent line equation}
\begin{equation}
	y-y_1 = m(x-x_1) 
\end{equation}

\noindent
\text{Definition of euler's number}
\begin{equation}
	e = \lim_{b \to \infty} (1+\frac{1}{b})^b 
\end{equation}

\noindent
\text{Derivative of constant}
\begin{equation}
	\frac{d}{dx} c = 0 
\end{equation}

\noindent
\text{Derivative of addition}
\begin{equation}
	\frac{d}{dx} [f(x)+g(x)] = f'(x)+g'(x) 
\end{equation}

\noindent
\text{Derivative of function multiplied by a constant}
\begin{equation}
	\frac{d}{dx} c * f(x) = c * f'(x) 
\end{equation}

\section{Identities}
Trig Identities
\begin{equation}
	\tan \theta = \frac{\sin \theta}{\cos \theta}
\end{equation}

\noindent
\text{Derivative of }
\begin{equation}
	\frac{d}{dx} 
\end{equation}

\noindent
\text{Derivative of cosine}
\begin{equation}
	\frac{d}{dx} \cos x = -\sin x
\end{equation}

\noindent
\text{Derivative of sine}
\begin{equation}
	\frac{d}{dx} \sin x = \cos x
\end{equation}

\noindent
\text{Derivative of tangent}
\begin{equation}
	\frac{d}{dx} \tan x = \sec^2 x
\end{equation}

\noindent
\text{Derivative of cotangent}
\begin{equation}
	\frac{d}{dx} \cot x = \csc^2 x
\end{equation}

\noindent
\text{Derivative of secant}
\begin{equation}
	\frac{d}{dx} \sec x = \sec x \tan x
\end{equation}

\noindent
\text{Derivative of cosecant}
\begin{equation} 
	\frac{d}{dx} \csc x = -\csc x \cot x
\end{equation}

\noindent
\text{Derivative of inverse sine}
\begin{equation}
	\frac{d}{dx} \sin^{-1} x = \frac{1}{\sqrt{1-x^2}}
\end{equation}

\noindent
\text{Derivative of inverse tangent}
\begin{equation}
	\frac{d}{dx} \tan^{-1} x = \frac{1}{1+x^2}
\end{equation}

\noindent
\text{Pythagorean theorem with sin/cos}
\begin{equation}
	\sin ^{2}(x)+\cos ^{2}(x)=1
\end{equation}

\noindent
\text{Rules of logarithms}
\begin{equation}
	\begin{aligned}
		log_b (b^x) = x
		b^{(log_{b}\ x)} = x
		log_{b}\ a^n = n \cdot log_b\ a
	\end{aligned}
\end{equation}

\noindent
\text{Rules of exponents}
\begin{equation}
	\begin{aligned}
		\left(\frac{a}{b}\right)^n = \frac{a^n}{b^n} \\
		(x^a)^b = x^{ab} \\
		x^a \cdot x^b = x^{a + b} \\
		x^{-n} = \frac{1}{x^n} \\
		\frac{a}{x} \cdot \frac{b}{x} \cdot \frac{c}{x} \cdot ... = (a \cdot b \cdot c \cdot ...)^{1/x} \\
	\end{aligned}
\end{equation}

\end{document}
